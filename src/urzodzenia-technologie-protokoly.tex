
\chapter{Wykorzystane narzędzia, technologie i~protokoły}

\section{Urządzenia wykorzystywane w~projekcie}
\subsection{ESP32}
ESP32 to jednoukładowy mikrokontroler, zaprojektowny i~produkwany przez firmę Espressif Systems. Jego najważniejsze cechy to:
\begin{itemize}
    \item energooszczędny procesor RISC o częstotliwości do 240 MHz
    \item 520 kB pamięci SRAM
    \item WiFi 802.11 b/g/n
    \item Bluetooth
    \item liczne interfejsy cyfrowe i analogowe, w~tym:
          \begin{itemize}
              \item UART
              \item I2C
              \item SPI
              \item I2S
              \item CAN
              \item ADC
              \item DAC
              \item PWM
              \item Ethernet MAC
              \item USB 2.0
          \end{itemize}
    \item \dots
          % \item 32 MB pamięci flash
\end{itemize} \cite{ESP32:datasheet} \\
Powstało wiele wersji tego układu, rózniące się m.in. szybkością procesora, ilością pamięci flash, ilością pinów, ilością interfejsów cyfrowych i analogowych, a także możliwością pracy w trybie bezprzewodowym (WiFi) lub przewodowym (Ethernet)\cite{ESP32:socs}.
Najczęsciej układ te wykorzystywane różnych projektach IoT, zarówno jako czujniki, jak i~serwery.\cite{ESP32:datasheet}
\subsection{Rasberry Pi Pico}

\subsection{STM32}
Rasberry Pi Pico to płytka z mikrokontrolerem RP2040, zaprojektowana i~produkwany przez firmę Raspberry Pi Foundation. Charakteryzuje się ona dwurdzeniowym procesorem ARM Cortex-M0+ o częstotliwości 133 MHz, 264 kB pamięci SRAM oraz 2 MB pamięci flash. Płytka posiada również wiele interfejsów cyfrowych i analogowych, w~tym:
\begin{itemize}
    \item UART
    \item I2C
    \item SPI
    \item I2S
    \item ADC
    \item DAC
    \item PWM
    \item USB 1.1
\end{itemize}
~\cite{PICO:datasheet,PICO:doc}
Płytka

\section{Języki programowania i technologie}
\subsection{\texttt{C++} for Arduio}
---
\subsection{\texttt{C} for STM32}
---
\subsection{\texttt{MicroPython} for Rasberry Pi Pico}
---
\subsection{\texttt{Python} for MQTT}

\section{Protokoły komunikacyjne}
---
\subsection{\texttt{MQTT}}
---
\subsection{\texttt{LoRa}}
---
\subsection{\texttt{HTTP}}
---
\section{Bazy danych i pozostałe technologie}
\subsection{InfluxDB 2}
---
\subsection{Docker}
---
\subsection{PlatformIO}
---
