
\chapter{Wykorzystane narzędzia, technologie i protokoły}

\section{Urządzenia}
\subsection{ESP32}
ESP32 to jednoukładowy mikrokontroler, zaprojektowny i produkwany przez firmę Espressif Systems. Jego najważniejsze cechy to:
\begin{itemize}
    \item energooszczędny procesor RISC o częstotliwości do 240 MHz
    \item 520 kB pamięci SRAM
    \item WiFi 802.11 b/g/n
    \item Bluetooth
    \item liczne interfejsy cyfrowe i analogowe, w tym:
          \begin{itemize}
              \item 2x UART
              \item 2x I2C
              \item 2x SPI
              \item 2x I2S
              \item 2x CAN
              \item 2x ADC
              \item 2x DAC
              \item 2x PWM
              \item 2x LED PWM
              \item 2x Hall sensor
              \item 2x SDIO
              \item 2x Ethernet MAC
              \item 2x USB 2.0
          \end{itemize}\cite{ESP32:datasheet}
          % \item 32 MB pamięci flash
    \item \dots
\end{itemize}
Powstało wiele wersji tego układu, rózniące się m.in. szybkością procesora, ilością pamięci flash, ilością pinów, ilością interfejsów cyfrowych i analogowych, a także możliwością pracy w trybie bezprzewodowym (WiFi) lub przewodowym (Ethernet)\cite{ESP32:socs}.
\subsection{Rasberry Pi Pico}
---
\subsection{STM32}
---

\section{Języki programowania i technologie}
\subsection{\texttt{C++} for Arduio}
---
\subsection{\texttt{C} for STM32}
---
\subsection{\texttt{MicroPython} for Rasberry Pi Pico}
---
\subsection{\texttt{Python} for MQTT}

\section{Protokoły komunikacyjne}
---
\subsection{\texttt{MQTT}}
---
\subsection{\texttt{LoRa}}
---
\subsection{\texttt{HTTP}}
---
\section{Bazy danych i pozostałe technologie}
\subsection{InfluxDB 2}
---
\subsection{Docker}
---
\subsection{PlatformIO}
---
