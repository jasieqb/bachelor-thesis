\listof{lstfloat}{Spis listingów} % jeśli są listingi
\addcontentsline{toc}{chapter}{Spis listingów}

% \listoftables{} % jeśli są tabele
% \addcontentsline{toc}{chapter}{Spis tabel}

\listoffigures{} % jeśli są rysunki
\addcontentsline{toc}{chapter}{Spis rysunków}
\printbibheading{}
\printbibliography[heading=subbibliography,nottype=online,title={Artykuły}]{} % wydruk bibliografii
\printbibliography[heading=subbibliography,type=online,title={Odnosniki w sieci}]{} % wydruk bibliografii
\addcontentsline{toc}{chapter}{Bibliografia} % też ręczne dodanie do spisu treści, jak Wstęp
% \bibliographystyle{plain}
% \begin{thebibliography}{99}
%     \bibitem{bib:a} aaaaaaaa
%     \bibitem{bib:b} bbbbbbbb
%     \bibitem{bib:esp32-all-socs} \href{https://www.espressif.com/en/products/socs}(dostęp: 15.04.2023)
% \end{thebibliography}
