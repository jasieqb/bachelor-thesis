
\chapter*{Wstęp} % z gwiazdką, więc bez numerka...
\addcontentsline{toc}{chapter}{Wstęp} % ...ale w spisie treści ma być
W dzisiejszych czasach ważne jest zbieranie różnorodnych informacji, z różnych, często trudno dostępnych miejsc. W~tym celu wykorzystywane są urządzenia, które zbierają informacje o~otoczeniu, a~następnie przekazują je do systemu, który następnie je przetwarza. Tego rodzaju czujniki mogą być umieszczone w~trudno dostępnych miejscach, na przykład na wysokich budynkach czy w~głębokich studniach. Nie ma tam możliwości poprowadzenia kabli, dlatego konieczne jest zastosowanie bezprzewodowych technologii komunikacyjnych. Jedną z~takich technologii jest LoRa, która pozwala na komunikację na duże odległości, przy niskim poborze mocy.

Celem niniejszej pracy jest stworzenie systemu zbierającego dane z~rozproszonych czujników z~wykorzystaniem technologii LoRa oraz ich zapisywanie w bazie danych w~celu dalszego przetworzenia.
W ramach pracy został stworzony system zbierający dane z różnych czujników, takich jak termometry~czy higrometry.
Informacje są zbierane przez autorskie punkty dostępowe LoRa, które następnie przekazują je do serwera.
Serwer przechowuje dane w bazie danych InfluxDB 2, która został wybrana do zastosowań IoT.

%TODO: fix this
W rozdziale pierwszym zostały opisane technologie, protokoły i urządzenia wykorzystane w projekcie.
W kolejnym rozdziale przedstawiono istniejące rozwiązania, zarówno przemysłowych jak i naukowych. Zostały one również porównane z rozwiązaniem zaproponowanym w ramach pracy.
W rozdziale trzecim zostały opisane założenia projektowe oraz szczegółowa implementacja systemu.
W czwartym rozdziale został opisany proces wdrożenia systemu oraz przeprowadzone testy. W ramach testów została sprawdzona wydajność systemu. Zostały również przedstawione wnioski z testów.
W piątym rozdziale zostały przedstawione wnioski z pracy oraz możliwości rozwoju systemu.
