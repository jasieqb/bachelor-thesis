\chapter{Założenie i Implementacja}
W poniższym rozdziale zaprocentowano założenia, potrzeby~i projekt projektu Sieci Meash na bazie LoRa

\section{Podstawowe cele sieci}
System ma kilka podstawowych założeń:
\begin{itemize}
    \item Jeden centralny punkt gromadzenia danych
    \item Zapisywanie danych do bazy danych szeregów czasowych, w celu ich dalszego przetwarzania
    \item Zbieranie danych z dużego obszaru
    \item Niezależność od istniejących metod przesyłu danych (WiFI, Sieci komórkowe, Łączność satelitarna)
    \item Niezależność od platformy sprzętowej
    \item Zapewniać możliwie dużą dostarczalność pakietów
    \item Dane czasowe nie muszą być super dokładne (dopuszczalne są drobne opóźnienia)
    \item Węzły sieci tylko wysyłają dane, same nie konsumują przychodzących wiadomości
\end{itemize}

\section{Części sieci}
W celu uzyskania wyżej wspomnianych założeń, zaproponowano system złożonych z kilku części:
\subsection{Węzły sieci}
Węzły sieci to urządzenia wyposażone w moduł LoRa i odpowiednie oprogramowanie pozwalające na pełną obsługę sieci. Gdy węzeł odbierze wiadomość, sprawdza jej poprawność i rozsyła ja dalej w celu zapewnienia jak największego zasięgu i wystarczalności

Urządzenie to może być również wyposażone w różnego rodzaju czujniki, które dostarczają danych bazy danych.

\subsection{Stacja przekaźnikowa}
Stacja przekaźnikowa to urządzenie wyposażone zarówno w moduł LoRa jak i moduł umożliwiający komunikację z siecią Internet (np. moduł WiFI lub moduł bazy Ethernet).

Urządzenie to odbiera przychodzące wiadomości LoRa i przesyła je do brokera wiadomości

\subsection{Broker wiadomości}
Broker wiadomości to program, działający na komputerze mającym dostęp do sieci, umożliwia on wydajną komunikację pomiędzy Stacją Przekaźnikową a Bazą Danych

\subsection{Baza danych}
Baza Danych umożliwiająca zapisywanie sporej ilości danych, uwzględniając również ich czas (baza danych szeregów czasowych).
O zapisy danych z brokera wiadomości do bazy danych dba osobny program, który powinien sprawdzać również poprawność tych wiadomości, jak i dbać o to by nie zapisywać powtórzonych wiadomości

\section{Wiadomości}
Każda z wiadomości przesyłanych za pomocą tych sieci powinna mieć formę jak zaprezentowano na listingu~\ref{lst:packet_format}

Zawiera ona pola:
\begin{itemize}
    \item \texttt{ttl} — [Ang. time to live — czas życia] Wartość określająca maksymalną liczbę skoków pomiędzy węzłami sieci. Domyślnie wynosi 10, może zostać wydłużona w zależności od wielkości planowanej sieci
    \item \texttt{m\_id} — UUID \cite{RFC:uuid} wiadomości, gwarantujący niepowtarzalności tej wiadomości. Ułatwia również jej dalsze przetwarzanie
    \item \texttt{d\_id} — numer identyfikacyjny urządzenia z którego pochodzi wiadomość
    \item \texttt{values} — słownik zawierający dane z urządzenia, do zapisania w bazie
\end{itemize}

\begin{lstfloat}[b]
    \lstset{language=JavaScript}
    \begin{lstlisting}[frame=single]
{
    "d_id": "id_233",
    "values": {
        "temp": "21",
        "hum": "50",
        "press": "1000",
        "light": "100",
        "co2": "1000",
        "pm25": "10",
        "pm10": "20"
    },
    "ttl": 10,
    "m_id": "eaa17a7b-9388-43b6-9310-731c942fc6b9"
}          
\end{lstlisting}
    \caption{Przykładowa wiadomość przesyłana przez system}\label{lst:packet_format}
\end{lstfloat}

\section{Obsługa protokołu}

\subsection{Działanie węzłów}
Wiadomości generowane są przez węzły sieci, zawierając wszystkie niezbędne pola (wymienione wyżej) i odczyty z czujników zamieszczonych na węźle. Następnie zostaje ona rozesłana do wszystkich węzłów w zasięgu (broadcasting).

Węzeł odbierając wiadomość, sprawdza jej poprawność (czy jest odpowiednio sformatowana, czy zawiera wszystkie potrzebne pola), i jeżeli wiadomość jest poprawna,a pole \texttt{ttl} jest większe od 0 rozsyła wiadomość dalej.
Sprawdzanie wiadomości odbywa się by wyeliminować wiadomości niepoprawne z sieci.

% TODO wstawić rysunek 

\subsection{Działanie przekaźnika}
Przekaźnik odbiera wiadomości, i przesyła je do brokera wiadomości. Nie sprawdza poprawności wiadomości by zapewnić maksymalną wydajność i niezawodność.

\subsection{Działanie bazy danych i programu zapisującego dane}
Program pobiera kolejne wiadomości od brokera i przetwarza je w kolejności:
\begin{enumerate}
    \item Sprawdzenie poprawności wiadomości
    \item \label{itm:powtorki} Sprawdzenie czy wiadomość nie została już sprawdzona (na podstawie pola \texttt{m\_id})
    \item Zapisanie danych ze słownika \texttt{values} do bazy danych i przyporządkowanie ich do \texttt{d\_id}
    \item Zapisanie w pamięci operacyjnej \texttt{m\_id}, potem do wykorzystania w~kroku \ref{itm:powtorki}
\end{enumerate}

\section{Implementacja}
\subsection{Implementacja przykładowych węzłów sieci}
W wyniku pracy nad systemem zbierania danych przygotowano implementację węzłów sieci w wykonrzystaniem 3 platform sprzętowych:
% TODO sprawdzić jakie dokładnie to modele
\begin{itemize}
    \item Rasberry Pi Pico +
    \item ESP32 TTYGO
    \item STM32 WL\dots
\end{itemize}
